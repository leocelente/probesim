\documentclass[a4paper, 12pt]{article}
\usepackage{amssymb, amsmath, graphicx, xcolor}

\usepackage{float, listings}
\pagecolor[rgb]{0.,0.,0.}
\color[rgb]{1,1,1}
\renewcommand{\baselinestretch}{1.5}
\usepackage{courier} 
\lstset{basicstyle=\footnotesize\ttfamily}
\lstset{frame=shadowbox} 

\usepackage{multirow}
\usepackage{geometry}
\geometry{a4paper,total={7in, 9in}}

\usepackage[T1]{fontenc}
\usepackage[utf8]{inputenc}
\usepackage[brazilian]{babel}
\usepackage{helvet}

\title{Equações do Simulador de Sondas Atmosféricas}
\author{Leonardo Celente}
\begin{document}
\maketitle
\tableofcontents
\pagebreak
\section{Atmosfera Padrão}
\subsection {Temperatura}

\begin{equation}
  T_{ar}(h)  = \left\{
  \begin{array}{ll}
    -x & \quad x \leq 0 \\
    -x & \quad x \leq 0 \\
    x  & \quad x \geq 0
  \end{array}
  \right.
\end{equation}


\subsection {Pressão}

\begin{equation}
  p_{ar}(h)  = \left\{
  \begin{array}{ll}
    -x & \quad x \leq 0 \\
    -x & \quad x \leq 0 \\
    x  & \quad x \geq 0
  \end{array}
  \right.
\end{equation}


\subsection {Densidade}
\begin{equation}
  \rho_{ar}(h) = \frac{p_{ar}(h)}{286.9 \, T_{ar}(h)}
\end{equation}

\section {Balão}
\subsection {Volume}

Partindo da Lei dos Gases:
% vol: float = m_gas * R * temperature / pressure / molar_mass_he

\begin{equation}
  V (h) =   T(h) R \frac{m_{gas}}{M_{he}}
\end{equation}

Assumindo que a temperatura do balão é a mesma do ar, então $T(h) = T_{ar}(h)$.
Também assumindo que a pressão do balão também está em equilibrio $p(h) = p_{ar}(h)$.

\subsection {Peso}
Bem simples:
\begin{equation}
  W = -\left( m_{balao} + m_{payload} + m_{helio} \right) g
\end{equation}

\subsection {Empuxo}
\begin{equation}
  E = g \, \rho_{ar}(h) V_{balao} \,
\end{equation}
\subsection {Arrasto}
\begin{equation}
  D = -\frac{1}{2} \, C_d \, A(h) \, \rho_{ar}(h) \,  \lvert \vec{v} \rvert \cdot \vec{v}
\end{equation}
Onde $A(h)$ é calculada ainda assumindo um balão esférico.
\begin{gather}
  r(h) = \frac{3\pi}{4} \sqrt[3]{V(h)} \\
  A(h) = \pi \left[r(h)\right]^2
\end{gather}
\subsection {Aceleração}
\begin{equation}
  \dot h = a = \frac{E + W + D}{m_{total}}
\end{equation}

\end{document}
